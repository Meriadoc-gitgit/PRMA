\documentclass{article}
\usepackage{graphicx} % Required for inserting images

\title{Kick-off meeting}
\author{Vu Hoang Thuy Duong, Jeannne Bonnaventure, Samy Bishay}
\date{16 Janvier 2024}

\begin{document}

\maketitle

\section{Informations générales}
UE à 6 ECTS corresponds à relativement 60h de travail
\subsection{Soutenance finale}
\begin{enumerate}
    \item Surement après les examens du semestre
    \item Chaque groupe assiste à la présentation des autres groupes
    \item Soutenance composée de 20 minutes de présentation et 20 minutes de questions
    \item Note de soutenance pas obligatoirement la même pour tous les membres du groupes si déséquilibre de travail/investissement
\end{enumerate}
\subsection{Informations supplémentaires}
\begin{enumerate}
    \item Possibilité de faire la recherche en anglais
    \item Réunions régulières \textit{1 par semaine} pour gérer l'avancement du projet, poser des questions...
\end{enumerate}
\section{Pratiques}
\subsection{Organisation}
\begin{enumerate}
    \item Fortement conseillé de faire un retro-planning avec une liste de tâches
    \item Gestion du temps en autonomie
    \item La charge de travail étant importante il est nécessaire de tirer avantage du travail en groupe et se repartir correctement les tâches
    \item noter attentivement ce que l’on a fait pour une tâche : avancement, temps, actions, outils...
    \item des comptes rendus doivent être envoyés à notre enseignant après chaque réunion le but étant de tenir un historique de l’avancement ainsi que s’assurer de la compréhension commune des réunions, ils doivent contenir une liste d’éléments factuels ainsi qu’un relevé de décisions (attribution des tâches, objectifs, questions en cours..)
    \item nécessaire de connaître et s’organiser en connaissance des deadlines
    \item des ordres du jour sont à envoyer au minimum 24h avant une réunion ils doivent contenir des questions et nous devons être chacun en mesure d’expliquer où en est le projet
    \item les rapports intermédiaires sont à faire relire en avance par notre encadrant avant l’envoie à Mr Kordon
    \item lors des réunions être concret sur ce que l’on dit et savoir dire je ne sais pas
    \item comprendre profondément le sujet, les enjeux et les problématiques
    \item construire son propre chemin de réflexion, le suivre et être consistant tout le long du semestre dans son travail
    \item il faut être capable de justifier tous les éléments du projet même ceux qui sont fournis
    \item notre code doit pouvoir être compréhensible, maintenu dans le temps et fonctionner sur n’importe quel machine/environnement
\end{enumerate}

\subsection{Traces et Documents}
\begin{enumerate}
    \item il est demandé de faire une bibliographie
    \item le repository doit contenir un readme
    \item tout doit être déposé dans un repository : source, code, informations.. le but est de retrouver facilement les informations/items
    \item le code doit pouvoir être compréhensible et maintenu dans le temps
    \item ne pas mettre d’artefacts dans le repository uniquement les sources (certaines exceptions par exemple si l’on utilise un fichier .o dont on est pas à l’origine

\end{enumerate}

\subsection{Outils}
\begin{enumerate}
    \item Github
    \item Bimer
    \item LaTeX
    \item Pretty Printer
\end{enumerate}
\end{document}
